\chapter{Conclusion}
%After applying the concept of justification from \cite{Justification} on PO and POCL plans.
%The effectiveness of justification algorithm on PO and POCL plans was tested on test sets with more than 400\,000 plans.
%The Table 4.1-4.4 show the resulting optimizations, with the
%greedy justification algorithms as the top perfoming algorithms by reaching plan step ratios of $\approx$ 77\%. 
%Which approves the correlation of justification strength with the algorithm running time see Figure 3.1. 
%Furthermore the effectiveness over differently sized plans has also been evaluated.
%The results are displayed in Table 4.5 to 4.8 which showed the weakness of well justification algorithm on small plans.
%By comparing Table 4.1 and 4.2 with Table 4.3 and 4.4 the rather small impact of the plan step ordering choise comes apparent.
%Another important observations is the superiority of PO optimization over POCL optimization due to increased ability to change plans. 

%As a second optimization method the rearrangement of ordering constraints in order to reduce the makespan has been evaluated.
%For this purpose the hybrid planning system PANDA\cite{Panda} has been used to create a makespan optimal ordering for the plan steps of the tested plans.
%Table 6.1 and 6.2 are showcasing the results of this evaluation. 
%The most noteable observations is the computation time of 922 minutes for 7429 plans with 2198 optimized plans.
%All the justification algorithms achieve better optimizations in less time.
%Therefore rearrangement can be used as way to further optimize plans with already optimal plan steps.
%For general optimization purposes the justification algorithms with their ability to remove plan steps proved to be more effective. 

In this work the justification algorithms which were featured in \cite{Justification} have been deployed on PO and POCL plans,
this led to two versions of the well and greedy justification algorithms, a PO and a POCL version.
These have been tested on a set of plans with more than 400\,000 plans.
The tests were able to prove that using a stronger justification type will yield better optimization results,
while also using a significantly higher computation time.
This confirms the justification hierarchy that has been described in \cite{Justification}.
Another important observation is the superiority of PO optimization over POCL optimization due to increased ability to change plans. 
The PO algorithms also had higher computation times,
which results in another tradeoff between computation time and optimization effectiveness. 
The combination of these factors leads to the following hierarchy between the four algorithms.

\begin{tabular}{|l|l|}
    \hline
    Greedy justification on PO plans&better optimization\\
    \cline{1-1}
    Greedy justification on POCL plans& $\uparrow$\\
    \cline{1-1}
    Well justification on PO plans & $\downarrow$\\
    \cline{1-1}
    Well justification on POCL plans &lower computation time\\
    \hline
\end{tabular}
\newline


Additionally, a way to reduce the makespan of a plan by rearranging its ordering constraints has been tested.
For this the initial ordering constraints were removed and the resulting incomplete plan would be transformed in a hybrid planning problem,
which would be solved with the hybrid planning system PANDA \cite{Panda} by inserting ordering constraints that result in an optimal makespan. 
This method proved to have much higher computation time than justification algorithms.
Also all the justification algorithms were better in optimizing the makespan of the tested plans.
Therefore, the rearrangement of ordering constraints can be used as a way to further optimize plans with already optimal plan steps.
For general optimization purposes the justification algorithms with their ability to remove plan steps proved to be more effective. 


