\chapter{Makespan Optimization}
There are more ways to optimize the makespan of a plan than just the removal of plan steps as it is done by the justification algorithms. 
In this chapter the goal is to minimize the makespan of a plan without changing any the set of plan steps. 
If no plan steps may be removed, the ordering of the plan steps has to be changed. 
\begin{figure}[h]
        \scheme{ATM}{0}{
        text={\textbf{0: go to ATM}},
        pres={},
        effs={money}
    }
    \scheme{friend}{0}{
        text={\textbf{1: call friend}},
        pres={},
        effs={friend called}
    }
    \scheme{borrow}{0}{
        text={\textbf{2: borrow money from friend}},
        pres={friend called},
        effs={money}
    }
    \begin{tikzpicture}
        \stage{20em}{27em}
        {effs = {}, eff length = 2em}
        {pres = {money},
        pre length = 2em,
        }{start}{end}
        \action{friend1}{friend={},
        body={above right = 4em and 3em of init.south},
        eff length = 6em,
        pre length = 3em}
        \action{ATM1}{ATM={},
        body={above right= 1/2em and 3em of friend1},
        eff length= 3em,
        pre length= 3em}
        \action{borrow}{borrow={},
        body={above right = 1/2em and 3em of ATM1},
        eff length = 3em,
        pre length = 6em}
        \ordering{borrow.north}{goal}
        \ordering{friend1.north}{ATM1.north}
        \ordering{init}{friend1.north}
        \ordering{ATM1.north}{borrow.north}
    \end{tikzpicture}
    \caption{Example: PO-plan with non-optimal ordering and a makespan of length 3}
\end{figure}
\begin{figure}[h!]
    \scheme{ATM}{0}{
        text={\textbf{0: go to ATM}},
        pres={},
        effs={money}
    }
    \scheme{friend}{0}{
        text={\textbf{1: call friend}},
        pres={},
        effs={friend called}
    }
    \scheme{borrow}{0}{
        text={\textbf{2: borrow money from friend}},
        pres={friend called},
        effs={money}
    }
    \begin{tikzpicture}
        \stage{20em}{27em}
        {effs = {}, eff length = 2em}
        {pres = {money},
        pre length = 2em,
        }{start}{end}
        \action{ATM1}{ATM={},
        body={below right = 2em and 20 em of init.north east},
        eff length= 3em,
        pre length= 3em}
        \action{friend1}{friend={},
        body={above right = 4em and 7 em of init.south east},
        eff length = 6em,
        pre length = 3em}
        \action{borrow}{borrow={},
        body ={right = 13em of friend1},
        eff length = 3em,
        pre length = 6em}
        \ordering{borrow.north}{goal}
        \ordering{init}{ATM1.north}
        \ordering{init}{friend1.north}
        \ordering{ATM1.north}{goal}
        \ordering{friend1.north}{borrow.north}
    \end{tikzpicture}
    \caption{Example: PO-plan with optimal ordering and a makespan of length 2}
\end{figure}

Figure 5.1 and 5.2 show how the ordering of plan steps can impact the makespan of a plan by using an example from Chapter 3.

For this approach we are using the hybrid planning system PANDA \cite{Panda}.
Hybrid Planning is a mixture of Hierarchical Task Network planning \cite{HTN} and POCL planning \cite{McAllester}.
Since, the goal is to find new ordering constraints for the given plan steps.
We define a hybrid planning problem that is solved by a 
PO plan with possibly new ordering constraints to the given plan steps.
 
A hybrid planning domain is defined by the tuple ($T_a$,$T_p$,$M$).
$T_a$ and $T_p$ are sets of primitive and abstract tasks and $M$ is a set of decomposition methods.
A primitive task is an unlabeled plan step i.e. an action.
An abstract task can be decomposed into a various amount of primitive and abstract tasks using decomposition methods.
Using abstract tasks can cause a variation in the resulting plan steps.
Therefore, to neither remove nor add additional plan steps, all given plan steps will be primitive tasks in our
hybrid planning domain.

A hybrid planning problem is defined by a hybrid planning domain and an initial Plan $P_{init}$.
It is solved by inserting ordering constraints and decomposition of abstract tasks, resulting in
a correct PO plan $P_{sol}$.

To find an optimal ordering of plan step, we need to discard all possibly suboptimal ordering constraints
in order to find better constraints.
Therefore, $P_{init}$  will not contain any ordering constraints. 

In conclusion the hybrid planning problem for reordering the PO plan $\Pi=(PS,\prec)$ has to satisfy the following criteria.

\begin{itemize}
    \item $T_a=\emptyset$
    \item $M=\emptyset$
    \item $T_p=PS$ 
    \item $P_{init}=(PS,\prec)$ with $\prec=\emptyset$ and $PS=\{init,goal\}$
\end{itemize}

Since, there are no abstract tasks PANDA solves this Hybrid planning problem with just the insertion of ordering constraints.
PANDA is also able to run on an A-star heuristic that minimizes the makespan.
In this configuration PANDA will create ordering constraints that will have the minimal possible makespan for 
the given plan steps.
The new plan will have the same plan steps as the old plan while also having makespan optimal ordering constraints.

