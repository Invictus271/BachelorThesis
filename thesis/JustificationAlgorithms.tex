\chapter{Justification Algorithms}
The basic idea behind Justification is to define a criterion for a plan step that describes its necessity for a plan. 
Different kinds of justification have been formalized and applied in algorithms by Fink and Yang
\cite{Justification}. These algorithms deployed on PO and POCL plans can find unjustified plan steps that can be removed
in order to create smaller, less complex plans. Smaller plans are in general more useful since they contain less \enquote{useless} plan steps
and are therefore easier and faster to execute, so the goal is to remove as many plan steps as possible. 
For some plans the amount of unjustified plan steps can vary on the justification type.
The algorithms can be sorted by their justification strength. A stronger algorithm can remove all plan steps that can be removed
by a weaker algorithm and possibly more. However, the runtime of the related algorithm also increases with the justification strength. 
There is a tradeoff between the amount of removed plan steps and the computation time.

\begin{figure}[h]
    \begin{tabular}{|l|l||l|}
        \hline
        \textbf{kind of justification} & \textbf{calculation time} & stronger justification\\
        \cline{1-2}
        greedy justification & $\mathcal{O}(P\cdot\vert \Pi \vert ^{5}] )$ & $\uparrow$\\
        \cline{1-2}
        well justification & $\mathcal{O}(P\cdot\vert \Pi \vert ^{4}] )$ & $\downarrow$\\
        \cline{1-2}
        backward justification &$\mathcal{O}(E\cdot\vert \Pi \vert ^{2}] )$ & weaker justification\\
        \hline
    \end{tabular}
    \caption{Three different types of justification and the runtime of its related algorithms on a PO plan $\Pi$,
    with $\vert \Pi \vert$ being the number of plan steps, $E$ the number of effects $E=\sum_{ps \in \Pi} \vert e^+(action(ps))\vert$ 
    and $P$ the number of preconditions $P=\sum_{ps \in \Pi} \vert p(action(ps))\vert$. Table taken from \cite{Justification}.}
\end{figure}
 
For each of the three justification types in figure 3.1 there is an algorithm for PO and POCL plans. The characteristics of these
algorithms will be discussed in the following sections.


\section{Backward Justification} 

\begin{Definition}
    \normalfont \textbf{Backward Justification \cite{Justification}}
    \textit{Let $\Pi$ be a PO plan that achieves the goal $g$. A plan step $ps \in \Pi$
is called backward justified if $\exists v\in e^+(action(ps))$ such that $ps$
possibly establishes $v$ either for the goal $g$ or for another backward justified plan step. 
We say that $ps$ possibly establishes a variable $v$ for a plan step $ps'$ in
a partially ordered plan $\Pi$ if $ps$ establishes $v$ for $ps'$ in at
least one linearization of $\Pi$.
}
\end{Definition}
\begin{algorithm}[H]
    \SetAlgoLined
    \KwData{let $\overline{\Pi}$ be some linearization of $\Pi$}
    \For{$ps$ := (last plan step of $\overline{\Pi})$ \KwTo (first plan step of $\overline{\Pi}$ )}{
        Justified=False\;
        \ForEach{$v \in e^+(action(ps))$}{
            \If{$\exists ps' \in \Pi$ s.t. $ps$ establishes $v$ for $ps'$ \textbf{or} $ps$ establishes $v$ for $g$}{
            \tcc{$ps$ is backward justified}
            Justified=True\;
        }
    }
    \If{Justified=False}{
        \tcc{$ps$ is not backward justified}
        remove ps from the plan $\Pi$\;
    }
}
\vspace{2em}
\caption{\cite{Justification} a backward justified subplan of a given plan}
\end{algorithm}
Algorithm 2 iterates through every variable in the add effect of each plan step and checks if these variables are established for other plan steps.
If there is a plan step that is not part of any establishment, the plan step is not justified in the given linearization of the plan.
Therefore, the plan step is also not justified in the related PO plan and can be removed.


As mentioned in section 2.2 a causal link $(ps_1,l,ps_2)$ means that $ps_1$ establishes $l$ for $ps_2$.
So every plan step in a POCL plan which is a producer of a causal link, does establish a variable for another plan step or goal in 
every linearization of the plan. So every plan step that is producer of a causal link is a backward justified plan step.
This the reason why backward justification is easy to establish with POCL plans. On the flip side most planners already create backward justified POCL plans,
therefore the related algorithm will most likely not optimize the plan. Above reasons show that the optimization of PO plans through 
backward justification is more effective than the optimization of POCL plans. 


\newpage
\begin{figure}[h]
    %\begin{tabular}{c|c|c}
    %  \toprule
    %state variables & initial state & goal description\\
    %\midrule
    %  \{money, friend called\} &  \{$\lnot$ money, $\lnot$ friend called\} & \{money\}\\
    %  \bottomrule
    %\end{tabular}
    %\newline
    %\vspace{2em}
    %\newline
    %\begin{tabular}{l|ccc}
    %    \toprule
    %    action name & precondition & add effects & delete effects \\
    %    \midrule
    %    go to ATM & $\emptyset$ & \{money\} & $\emptyset$ \\
    %    call friend & $\emptyset$ & \{friend   \link{init/eff/4}{pay1/pre/3}{edge[out=0,in=180,looseness=2,->]}called\} & $\emptyset$ \\
    %    borrow money from friend& \{friend called\} &\{money\}& $\emptyset$ \\
    %    \bottomrule
    %\end{tabular}
    %\newline
    %\vspace{2em}
    %\newline
    %\begin{tabular}{l|c}
    %    \toprule
    %    set of plan steps & ordering constraints \\
    %    \midrule
    %    0: go to ATM &  $1 \prec 2$ \\
    %    1: call friend & \\
    %    2: borrow money from friend & \\
    %    \bottomrule
    %\end{tabular}
    %\vspace{2em}
    %\newline
    \scheme{ATM}{0}{
        text={\textbf{0: go to ATM}},
        pres={},
        effs={money}
    }
    \scheme{friend}{0}{
        text={\textbf{1: call friend}},
        pres={},
        effs={friend called}
    }
    \scheme{borrow}{0}{
        text={\textbf{2: borrow money from friend}},
        pres={friend called},
        effs={money}
    }
    \begin{tikzpicture}
        \stage{20em}{27em}
        {effs = {}, eff length = 2em}
        {pres = {money},
        pre length = 2em,
        }{start}{end}
        \action{ATM1}{ATM={},
        body={below right = 2em and 20 em of init.north east},
        eff length= 3em,
        pre length= 3em}
        \action{friend1}{friend={},
        body={above right = 4em and 6 em of init.south east},
        eff length = 5em,
        pre length = 3em}
        \action{borrow}{borrow={},
        body ={right = 13em of friend1},
        eff length = 3em,
        pre length = 3em}
        \ordering{borrow.north}{goal}
        \ordering{init}{ATM1.north}
        \ordering{init}{friend1.north}
        \ordering{ATM1.north}{goal}
        \ordering{friend1.north}{borrow.north}
    \end{tikzpicture}
    \caption{Example: PO plan with two alternate linearizations}
\end{figure}

In PO plan there are no causal links that define the establishment of a variable. In two different linearizations 
of a PO plan one variable can be established by two different plan steps as can be seen in the example of Figure 3.2.
Plan step 1 establishes 'called friend' for plan step 2, and the plan steps 0 and 2 establish 'money' for the goal.
The PO plan in Figure 3.2 allows for three different linearizations: 
\begin{enumerate}
    \item $0 \to 1 \to 2 $
    \item $1 \to 2 \to 0$
    \item $1 \to 0 \to 2$
\end{enumerate}
In the first linearization plan step 0 is not backward justified, because plan step 2 establishes also the variable \enquote{money} but later.
In the second linearization plan step 1 and 2 are both not backward justified. The algorithm first removes plan step 2 
because plan step 0 is executed later. Then plan step 1 is removed because plan step 2 is deleted, so there is no precondition that needs the
establishment of \enquote*{friend called}. Notice the optimization of plan step 1 is only possible because the algorithm removes the plan steps from last to first.
The third linearization would be optimized like the first one because plan step 0 is executed before plan step 2. 
The example also showcases that backward justification can depend on the observed linearization. The algorithm deals with this problem 
by choosing a random linearization of the PO plan. The random decision of the linearization results in some non-deterministic justifications.
A strategy to get an optimal and deterministic result would be to observe every possible linearization and choose the optimization with the least
remaining plan steps. If there are plan steps that have no add effect, then they are always going to be removed, because there is no 
linearization where those plan step could possibly establish a variable. However, plan steps that have variables in their add effects that are already true
in the initial state cannot be removed by the backward justification algorithm.


Regarding the complexity of backward justification, for every variable $v$ which is part of an add effect 
the algorithm needs to check whether it is part of an establishment or not which equals the complexity $\mathcal{O}(E=\sum_{ps \in \Pi} \vert e^+(action(ps))\vert)$.
Then the algorithm needs to check for every plan step that is ahead, whether $v$ can be established for this plan step or not with the complexity $\mathcal{O}(\vert \Pi \vert)$.
Then for every possible establishment the algorithm needs to check, if there can be a plan step with $v$ as a delete effect which also has the complexity of $\mathcal{O}(\vert \Pi \vert)$.
Therefore, the overall running time of the algorithm is $\mathcal{O}(E\cdot\vert \Pi \vert ^{2}] )$.
\section{Well Justification}

\begin{Definition}
    \normalfont \textbf{Well Justification \cite{Justification}}
    \textit{A plan step $ps_i$ in a total-order plan $\overline{\Pi}$ is called well-justified if $\exists v \in e^+(action(ps))$
such that $ps_i$ establishes $v$ for some plan steps or
for the goal $g$, and $v$ does not hold before $ps_i$, that is
$v \notin s_{i-1}$. A plan step in a partially ordered plan is called well-justified if it is 
well-justified in at least one linearization of the plan.
}
\end{Definition}
A difference between well justification and backward justification is that a variable $v$ that is already true in a state $s$ 
can no longer be established by other plan steps. If the variable $v$ is true either in the initial state or through the add effect of $ps_1$ 
and there are two plan steps $ps_2$ $ps_3$ with $ps_1 \prec ps_2 \prec ps_3$ and $ps_2$ establishes $v$ for $ps_3$. $ps_2$ would be backward justified but not 
well justified, given that there is no plan step $ps_4$ with $v \in e^-(action(ps_3))$ that can be executed before $ps_2$.
The backward justifications criteria are included in the definition of well justification alongside the above explained criterion.
Therefore, every well justified plan step is also backward justified, but not every backward justified plan step is 
well justified.
Because every plan step is only well justified if it establishes some variable that was not established before (or is true in the initial state) the
removal of a well justified plan step will lead to the incorrectness of the plan. Therefore, the following lemma holds.

\newtheorem{Lemma}{Lemma}
\begin{Lemma}
    \cite{Justification} \textit{A plan step is well-justified if and only if we
    cannot remove it from the plan without violating correctness of the plan.}
\end{Lemma}
  The next theorem follows directly from the lemma.
\newtheorem{theorem}{Theorem}
\begin{theorem}
    \cite{Justification} \textit{A plan is well-justified if and only if there
    is no plan step that can be removed without violating correctness of the plan.}
\end{theorem}
In order to find a well-justified subplan of a PO plan, algorithm 2 needs to be applied.
\newline
\vspace{1em}
\newline
\begin{algorithm}[H]
    \SetAlgoLined
    \KwData{let $\Pi$ be a PO plan}
    \Repeat{no plan step is removed durning the last execution of the loop}{
        \ForEach{$ps \in \Pi$}{
            \If{$\Pi$ without ps is legal and achieves the goal}{
                remove ps from $\Pi$\;
            }
        }
    }
    \vspace{2em}
    \caption{Finding a well justified subplan of a given PO plan}
\end{algorithm}

Algorithm 2 tries to remove all plan steps one by one and then checks whether the plan is still legal.
If a plan step gets removed, some establishments might be removed with it. 
Therefore, if a plan step is being removed the algorithm tries again to remove plan steps.
This leads to the removal of plan steps that only establish variables for not well justified plan steps.
This algorithm uses a similar non-determinism like the backward justification algorithm. Depending on the
choice of the plan step in line 2, different subplans will emerge. The choice is, in general, random and an optimal reduction of 
plan steps for a plan is achieved if every possible order of plan step choices is computed and the plan with the least amount 
of plan steps is selected. This random ordering also determines if a chain of plan steps can be removed. For example in Figure 3.2. plan step 2
needs a variable that is created by plan step 1, so if plan step 2 is being removed plan step 1 is also being removed.
If plan step 1 is attempted to be removed first the plan without it is not legal anymore,
because the precondition of plan step 2 is not satisfied anymore, therefore 2 needs to be eliminated first. This kind of ordering problem is already addressed in the
backward justification algorithm because the plan steps are being selected in a reversed order. One way to optimize the well justification algorithm would be 
to select the plan steps with the least plan steps ordered behind them first.


In order to execute algorithm 2, we need to be able to check the condition in line 3. This can be accomplished with algorithm 3.
\newpage

\begin{algorithm}[h]
    \SetAlgoLined
    \SetKwInOut{Input}{input}\SetKwInOut{Output}{output} 
    \Input{$\Pi$ a PO plan}
    \Output{True if $\Pi$ is correct, False otherwise}
    \ForEach{$ps_1 \in \Pi$}{
        \ForEach{$v \in p(action(ps_1))$}{
            established $\leftarrow$ False\;
            \ForEach{$ps_2 \in predecessors(ps_1)$}{
                \If{$\exists v \in e^+(action(ps_2))$ \textbf{and} $v \in p(action(ps_1))$}{
                    established $\leftarrow$ true\;
                    \ForEach{$ps_3 \in possiblePredecessors(ps_1) \cap possibleSuccessors(ps_2)$}{
                        \If{$v \in e^-(ps_3)$}{
                            established $\leftarrow$ False\;
                        }
                    }
                }
            }
            \If{established=False}{
                \Return False\;
            }
        }
    }
    \Return True \;
    \caption{Checking the correctness of a PO plan}
\end{algorithm}

Algorithm 3 checks for every precondition variable of every plan step if there is a predecessor plan step that establishes this variable.
Therefore, the algorithm needs to check if there is a predecessors with the variable in its add effects and if there could be a delete effect 
which destroys the establishment. 
For this computation Algorithm 3 uses the functions $predecessors$, $successors$, $possiblePredecessors$ and $possibleSuccessors$, that each return a set of plan step.
They are defined as follows.

\begin{itemize}
    \item predecessors($ps_1$)=$\{ps \in \Pi \setminus ps_1 \mid ps \prec ps_1 \}$
    \item successors($ps_1$)=$\{ps \in \Pi \setminus ps_1 \mid ps_1 \prec ps \}$
    \item possibleSuccessors($ps_1$)=$\{ps \in \Pi \setminus ps_1 \mid ps \notin predecessors\}$
    \item possiblePredecessors($ps_1$)=$\{ps \in \Pi \setminus ps_1 \mid ps \notin successors \}$
\end{itemize}

To compute these functions efficiently an ordering matrix needs be created. 
An ordering matrix is a squared matrix with the height/length of the amount of plan steps, filled with boolean values. For an ordering matrix $om$ the value 
$om[1][2]$ would state whether the ordering constraint $ps_1 \prec ps_2$ would hold.
As discussed in section 2.2 ordering constraints are transitive, to avoid redundant computation the ordering matrix can be replaced
by its transitive closure. This can be accomplished with the Floyd-Warshall Algorithm \cite{Floyd}. 
From such a transitive closure the sets $predecessors$, $successors$, $possiblePredeccessors$ and $possibleSuccesors$ can directly be computed.


The well justification algorithm can also be extended for POCL plans unlike the backward justification algorithm.
This is because of the way the algorithm selects the plan steps in line 2.
If a plan step that produces a causal link gets removed from the plan, it can still be a 
legal plan if there is another plan step that can also produce the same variable.
So if replacements for all removed causal links are added before checking the correctness of the plan an optimization can be accomplished.
Algorithm 6 searches such replacements for causal links that have been produced by a removed plan step.


In order to extend the well justification algorithm to POCL plans the condition in line 3 needs to be computed differently.
If we remove a plan step from a POCL plan that produced a causal link the plan is illegal because there would be an open precondition.
Therefore, replacements for such causal links need to be found without creating causal threats.
This problem is known as Remove and Repair and is addressed in \cite{RemoveRepair}.

\begin{algorithm}[H]
    \SetAlgoLined
    \KwData{let $\Pi$ be a POCL plan}
    \Repeat{no plan step is removed durning the last execution of the loop}{
        \ForEach{$ps \in \Pi$}{
            \If{replacementFound($\Pi$,$\{ps\}$)=(True,$\Pi'$) \textbf{and} checkCausalThreats($\Pi'$)=True}{
                $\Pi$ $\leftarrow$ $\Pi'$
            }
        }
    }
    \vspace{2em}
    \caption{Finding a well justified subplan of a given POCL plan}
\end{algorithm}

Algorithm 5 describes the extended well justification algorithm for POCL plans.
The condition in line 3 of algorithm 3 is replaced by two algorithms that are described in algorithm 5 and 6. 

\newpage
\begin{algorithm}[h]
    \SetAlgoLined
    \SetKwInOut{Input}{input}\SetKwInOut{Output}{output} 
    \Input{$\Pi=(PS,\prec,CL)$ a POCL plan, $PS'$ a set of plan steps that should be removed}
    \Output{(True,$\Pi$ with new Causal Links and without $PS'$) if replacements could be found, (False,$\Pi$) otherwise}
    oldCL $\leftarrow$ CL\;
    \ForEach{$ps_1 \in PS'$}{
    \ForEach{$cl \in CL$}
    {
        \If{$cl=(ps_1,v_1,ps_2)$}{
            \ForEach{$ps_3 \in possiblePredeccessors(ps_2)$}{
                \If{$\exists v_1 \in p(action(ps_3))$}{
                    replacementFound $\leftarrow$ True\;
                   % \ForEach{$cl \in CL$}{
                   %     \If{$\exists ps_4 \in possibleSuccesors(ps_3) \cap possiblePredeccessors(ps_2)$ \textbf{with} $v_1 \in e^-(action(ps_4))$}{
                   %         replacementFound $\leftarrow$ False\;
                   %     }
                   %}
                }
                \If{replacementFound=True}{
                    remove $cl$ from $CL$\;
                    add $cl':=(ps_3,v_1,ps_2)$ to $CL$\;
                }
                \Else{
                    \Return (False,$\Pi$ with oldCL)\;
                }
            }
        }
    }
    }
    remove $PS'$ from $\Pi$\;
    \Return (True,$\Pi$ with CL)\;
    \caption{Finding replacements for causal links: searchReplacements($\Pi$,$PS'$)}
\end{algorithm}

Algorithm 6 needs to replace all causal links $(ps_1,v,ps_2)$ where a removed plan step $ps_1$ was the producer with a causal link that has a new producer $ps_3$.
In order to accomplish this every $possiblePredeccessors(ps_2)$ is checked if they can establish $v$.


\begin{algorithm}[H]
    \SetAlgoLined
    \SetKwInOut{Input}{input}\SetKwInOut{Output}{output} 
    \Input{$\Pi=(PS,\prec,CL)$ a pocl plan}
    \Output{True, if $\Pi$ has no causal threat, False otherwise}
    \ForEach{$cl=(ps_1,v,ps_2) \in CL$}{
        \ForEach{$ps \in \Pi$}{
            \If{$v \in e^-(action(ps))$ \textbf{and} $ps \in possibleSuccesors(ps_1) \cup possiblePredeccessors(ps_2)$}{
                \Return False;
            }
        }
    }
    \Return True;
    \caption{Checking a POCL plan for causal threats: checkCausalThreats($\Pi$)}
\end{algorithm}

Algorithm 7 checks for every causal link $(ps_1,v,ps_2)$ if there is a plan step that can be ordered between $ps_1$ and $ps_2$ that 
has $v$ in its delete effects.

In the well justification algorithms (algorithm 2 and 4) there are $\mathcal{O}(\vert \Pi \vert^{2})$
attempts at removing plan steps from the plan. To check whether a PO plan is legal a running time 
of $\mathcal{O}(P\cdot\vert\Pi\vert^{2})$ ($P=\sum_{ps \in \Pi} \vert p(action(ps))\vert$) is needed.
For POCL plans the running time of searchReplacements and checkCausalThreats are both
 $\mathcal{O}(\vert CL \vert \cdot \vert \Pi \vert)$.
So the overall running time of the PO algorithm is $\mathcal{O}(P\cdot\vert\Pi\vert^{4})$ while 
the running time of the POCL algorithm is $\mathcal{O}(P \cdot \vert CL \vert \cdot \vert \Pi \vert^3)$.
\newpage


\section{Greedy Justification}

To check if a plan step $ps$ is greedily justified in a plan $\Pi$ algorithm 7 \cite{Justification} needs to be applied. 
If $ps$ is greedily justified the original plan, $\Pi$ is returned. If $ps$ is not greedily justified, a reduced plan $\Pi'$ is returned.
In $\Pi'$ $ps$ and possibly other also not greedily justified plan steps have been removed.


\begin{algorithm}[h]
    \SetAlgoLined
    \SetKwInOut{Input}{input}
    \SetKwInOut{Output}{output} 
     \Input{$\Pi$ a PO plan, $ps \in \Pi$ a plan step}
    \Output{A possibly reduced PO plan $\Pi$}
    remove $ps$ from $\Pi'$ \;
    \Repeat{$\Pi$ does not contain illegal plan steps}{
        Illegals := findIllegalsPO($\Pi$)\;
        EarliestIllegals :=  $\{ ps' \in \text{Illegals} \mid (\forall ps_1 \in \Pi): ps_1 \prec ps' \Rightarrow ps_1 \notin \text{Illegals}\}$ \;
        remove all plan steps of the set EarlieastIllegals from $\Pi$\;
    }
    \uIf{$\Pi$ still achieves the goal and is legal}{
        \Return{$\Pi$ with removed plan steps}\;
        \tcc{$\Pi$ is a legal subplan of the initial plan}
    }
    \Else{
        \Return{the initial plan $\Pi$}\;
        \tcc{$ps$ in the initial plan is greedily justified}
    }
        
    \caption{GreedyJustifyChecking($\Pi$,$ps$)}
\end{algorithm}


A plan step $ps$ in a PO plan $\Pi$ is considered illegal if a precondition $v \in p(action(ps))$ is not established.
This is the case if there is no $ps' \in \Pi$ that establishes $v$ for $ps$ according to Definition 1. 
Therefore, the set of illegals in PO plans is defined by


$illegals = \{ps \in \Pi \mid  \exists v \in p(action(ps)) \forall ps' \in \Pi \text{: ps' does not establishes v for ps}\}$ 


The set of illegal plan steps in a PO plan can be found with algorithm 8 which is a modified version of algorithm 3.
\newpage

\begin{algorithm}[H]
    \SetAlgoLined
    \SetKwInOut{Input}{input}
    \SetKwInOut{Output}{output}
    \Input{$\Pi$ a PO plan with possibly illegal plan steps}
    \Output{$illegals$,   a set containing all illegal plan steps of $\Pi$}
    $illegals$ $\leftarrow$ $\emptyset$ \;
    \ForEach{$ps_1 \in \Pi$}{
        \ForEach{$v \in p(action(ps_1))$}{
            established $\leftarrow$ False\;
            \ForEach{$ps_2 \in predecessors(ps_1)$}{
                \If{$\exists v \in e^+(action(ps_2))$ \textbf{and} $v \in p(action(ps_1))$}{
                    established $\leftarrow$ true\;
                    \ForEach{$ps_3 \in possiblePredecessors(ps_1) \cap possibleSuccessors(ps_2)$}{
                        \If{$v \in e^-(ps_3)$}{
                            established $\leftarrow$ False\;
                        }
                    }
                }
            }
            \If{established=False}{
               add $ps_1$ to $illegals$ 
            }
        }
    }
    \Return illegals \;ö
    \caption{Finding illegal plan steps in a PO plan: findIllegalsPO($\Pi$)}
\end{algorithm}

Well justification can also be applied to POCL plans. An illegal plan step in a POCL plan is defined by either missing  
a causal link to protect one of its precondition or by being the consumer of a threatened causal link.
Let $ps_1,ps_2 \in \Pi$ be two plan steps of a POCL plan. $ps_1$ is considered an illegal plan step if:
\begin{itemize}
    \item $\exists v \in p(action(ps_1))$ and $\lnot \exists cl \in CL$ with $cl=(ps_2,v,ps_1)$ or
    \item $\exists cl \in Cl$ with $cl=(ps_2,v,ps_1)$ and $\exists ps_3 \in possiblePredecessors(ps_1)\text{ }\cap \text{ }possibleSuccesors(ps_2)$
    with $v \in e^-(action(ps_3))$
\end{itemize}


In order to find the illegal plan steps of a POCL plan algorithm 9 needs to be applied.
\newpage

\begin{algorithm}[H]
    \SetAlgoLined
    \SetKwInOut{Input}{input}
    \SetKwInOut{Output}{output}
    \Input{$\Pi$ a POCL plan with possibly illegal plan steps}
    \Output{$illegals$ a set containing all illegal plan steps of $\Pi$}
    $illegals$ $\leftarrow$ $\emptyset$ \;
    \ForEach{$ps_1 \in \Pi$}{
        \ForEach{$v \in p(action(ps_1))$}{
            protected $\leftarrow$ False \;
            \ForEach{$cl \in CL$}{
                \If{$cl=(ps_2,v,ps_1)$}{
                    protected $\leftarrow$ True \;
                    \ForEach{$ps_3 \in possiblePredecessors(ps_1) \cap possibleSuccesors(ps_2)$}{
                        \If{$v \in e^-(ps_3)$}{
                            protected $\leftarrow$ False \;
                        }
                    }
                }
            }
            \If{protected=False}{
                add $ps_1$ to $illegals$\;
            }
        }
    }
    \caption{Finding illegal plan steps in a POCL plan: findIllegalsPOCL($\Pi$)}
\end{algorithm}
\newpage

With algorithm 9 and the algorithms from section 3.2 (algorithm 4 and 5) GreedyJustifyChecking
can be extended to POCL plans through algorithm 10.

\begin{algorithm}[H]

    \SetAlgoLined
    \SetKwInOut{Input}{input}
    \SetKwInOut{Output}{output} 
    \Input{$\Pi$ a POCL plan, $ps \in \Pi$ a plan step}
    \Output{A possibly reduced PO plan $\Pi$}
    remove $ps$ from $\Pi$ \;
    $PS'$ $\leftarrow$ $\emptyset$\;
    \tcc{$PS'$ tracks all removed plan steps}
    \Repeat{$\Pi$ does not contain illegal plan steps}{
        Illegals := findIllegalsPOCL($\Pi$) \;
        EarliestIllegals :=  $\{ \alpha' \in \text{Illegals} \mid (\forall \alpha_1 \in \Pi): \alpha_1 \prec \alpha' \Rightarrow \alpha_1 \notin \text{Illegals}\}$ \;
        remove all plan steps of the set EarlieastIllegals from $\Pi$\;
        add EarlieastIllegals to $PS'$\;
    }
    \If{replacementFound($\Pi$,$PS'$)=(True,$\Pi'$) \textbf{and} checkCausalThreats($\Pi'$)=True}{
        \Return{$\Pi'$ with removed plan steps}\;
        \tcc{$\Pi'$ is a legal subplan of the initial plan}
    }
    \Else{
        \Return{the initial plan $\Pi$}\;
        \tcc{$ps$ in the initial plan is greedily justified}
    }
    
    \caption{GreedyJustifyChecking($\Pi$,$ps$) extended to POCL plans}
\end{algorithm}


While GreedyJustifyChecking($\Pi$,$ps$) (algorithm 7 and 9) only examines one plan step, GreedyJustification($\Pi$) (algorithm 11) tries to 
remove all non greedy justified plan steps from $\Pi$.


\begin{algorithm}[H]
    \SetAlgoLined
    \SetKwInOut{Input}{input}
    \SetKwInOut{Output}{output} 
    \Input{$\Pi$ a PO or POCL plan}
    \Output{$\Pi$ a PO or POCL plan with possibly reduced plan steps}
    \ForEach{$ps \in \Pi$}{
        $\Pi'$ $\leftarrow$ GreedyJustifyChecking($\Pi$,$ps$)\;
        \If{$ps$ is not greedily justified in $\Pi$}
        {
            \Return{GreedyJustification($\Pi'$)}
        }
    }
    \Return{$\Pi$}
    \caption{GreedyJustification($\Pi$)}
\end{algorithm}


Greedy justification creates a stronger justification than well justification, but not a perfect one.
A perfect justified plan would be defined by creating a plan with no legal subplan. However, the problem of 
creating a perfect justified subplan is NP-complete. That is why greedy justification is used, it delivers an 
almost perfect justification in polynomial time. Figure 3.4 shows the big improvement to well and backward justification, the 
elimination of action cycles that destroy and reestablish variables. The actions $driveSchool$ and $driveBack$ destroy and 
reestablish the variable $at school$. However, they are both well justified because they establish preconditions for other plan steps.
They can only be removed using greedy justification.
\begin{figure}

    \scheme{driveSchool}{0}{
        text={\textbf{0: drive to school}},
        pres={at home},
        effs={at school}
    }
    \scheme{driveBack}{0}{
        text={\textbf{1: drive back}},
        pres={at school},
        effs = {
        {at home},
        {$\neg$at school}
  }
    }
    \scheme{driveSchool2}{0}{
        text={\textbf{2: drive to school}},
        pres={at home},
        effs={
            {at school},
            {$\neg$at home}
            }
    }
    \begin{tikzpicture}
        \stage{15em}{27em}
        {effs = {at home},eff length = 5em}
        {pres = {at school},
        pre length = 5em
        }{start}{end} 
        \action{driveSchool}{driveSchool={},
        body={above right = 2em and 5em of init.south},
        eff length= 5em,
        pre length=4em}
        \action{driveBack}{driveBack={},
        body={above right= 1/2em and 2em of driveSchool},
        eff length = 5em,
        pre length = 5em}
        \action{driveSchool1}{driveSchool2={},
        body={above right = 1/2em and 2em of driveBack},
        eff length= 5em,
        pre length = 5em}
        \ordering{init}{driveSchool.north}
        \ordering{driveSchool.north}{driveBack.north}
        \ordering{driveBack.north}{driveSchool1.north}
        \ordering{driveSchool1.north}{goal}
    \end{tikzpicture}
    \caption{In this PO plan all plan steps are well and backward justified. However the greedy justification algorithm
    would remove either the first or the last two plan steps.}
\end{figure}
The greedy justification algorithm also includes some non-deterministic decisions. Whether the first or the 
last two plan steps are going to be removed depends on line 2 of GreedyJustification($\Pi$). The iteration of \enquote{drive to school} which is chosen
first is going to be removed. As already mentioned for well justification, to get the best optimization every possible order of choices needs to
be computed and the solution with the least plan steps has to be selected.


The algorithm GreedyJustifyChecking for PO plans requires $\mathcal{O}(P \cdot \vert \Pi \vert ^{3})$ time.
For POCL plans GreedyJustifyChecking needs $\mathcal{O}(P \cdot \vert \Pi \vert \cdot \vert CL \vert^{2})$ time.
The algorithm GreedyJustification calls GreedyJustifyChecking at most $\vert \Pi \vert ^{2}$ times. 
So the overall complexity of the PO plan algorithm is $\mathcal{O}(P \cdot \vert \Pi \vert ^{5})$,
while the complexity for the POCL plan algorithm is $\mathcal{O}(P \cdot \vert \Pi \vert ^{3} \cdot \vert CL \vert^{2})$.

