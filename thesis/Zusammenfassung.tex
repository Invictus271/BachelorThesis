\chapter*{Abstract}


The purpose of planning is to get from an initial state to a goal state by ordering plan steps between them.
The creation of optimal plans demands a price in runtime performance therefore some planning algorithms 
create non-optimal plans.
In order to deal with those Eugene Fink and Qiang Yang presented Algorithms which are able to optimize a specific set of non-optimal plans.
They categorize plan steps inside of a plan as either justified or non-justified.
Every non-justified plan step can be removed without hurting the correctness of the plan.
In this work these algorithms will be applied on Partial-order (PO) and Partial-order-causal-link (POCL) plans.
Furthermore, the effectiveness of the algorithm will be evaluated to provide a proof of concept. 
While the justification algorithms optimize plans by removing plan steps, plans can also be optimized
by a reordering of the plan steps to reduce its makespan.
In this work both methods will be evaluated and compared by using metrics like reduction of plan steps, 
reduction of makespan, and necessary computation time. 