% author: Mario Schmautz
% modifications by: Pascal Bercher

% 
% Please read the "IMPORTANT NOTES" after the change log that
% warns you of some pitfalls in using the library
% 

% version: 1.05 (2018-02-15)
% 
% change log:
% 
% 1.00 --> 1.01
% - added version number and history of changes
% - moved required libraries from examples in here
% - added library fit, which enables fitting nodes together (useful for methods)
% 
% 1.01 -> 1.02
% - added "raise = .25 ex," to all text decorations (in actions), such that the
%   precondition and effect texts are not directly on the lines, but a bit above them
% 
% 1.02 -> 1.03
% - added parameters to stage command: instead of displaying 'start' and 'goal', those
%   strings can now be specified
% - added feature request list
% 
% 1.03 -> 1.04
% - added commands for adding ordering constraints
% 
% 1.04 -> 1.05
% - made the variable names of the stage command consistent: now they are "init" and "goal"
%   (before they were "start" and "goal"; also okay would have been "start" and "end", but
%    this synonym has not been added)
% - added a note explaining how to use the library accordingly
%  
% feature request list:
% - the start an end goal commands (cf. 1st entry, 1.03) should be optional with standard value
% - currently, one cannot draw actions with a single precondition/effect that has no content (text)
%   (some time later: really? I thought that's possible by specifying {} within the surrounding
%   parentheses... check!)
% - there should be a standard scheme for "almost empty" actions and just 3 parameters:
%   number of precs, number of effs, and the name. Maybe also optional arguments for height and width
%   (in many cases one uses these empty boxes, so providing extra commands for them would be appreciated)
% - the name of actions cannot be specified in math mode. this needs to be changed!!
% - actions sollte ein optionaler Parameter mitgegeben werden können, den man als tikz-Style definieren
%   kann; Beispiel: abstract/compound = rounded corners, double = \MyRed


% IMPORTANT NOTES: (usage)
% 
% - There is no documentation, please extract the usage from our examples; see "example.tex"
% - Note that when specifying math code ($...$) in the preconditions and effects, you need to
%   put it in an *additional* set of parentheses (in addition to the ones already there). Example:
%   pres = { {A} , {B} , {C} }, BUT:
%   pres = { {A{$_1$}} , {B{$_2$}} , {C{$_1$}} }, or simply
%   pres = { {{A$_1$}} , {{B$_2$}} , {{C$_1$}} }.
%   Otherwise, you will get strange compilation errors...
% - The same (see previous note) holds for other commands like \textbf{} -- they need to be
%   encapsulated in parentheses as well, e.g., when used as an action name.
% - If you want to use action names that use several lines (e.g., to put the 
%   parameters in different lines to save space), then use \shortstack{line-1\\...\\line-n}

\usetikzlibrary{shapes.geometric, calc, math, decorations.text,positioning, arrows.meta, fit}

\newcounter{aiplans@numeffs} % number of effects in the current action
\newcounter{aiplans@numpres} % number of preconditions in the current action

\tikzset{aiplans action body/.style = {draw}}
\tikzset{aiplans action limb/.style = {draw}}
\tikzset{aiplans emph/.style = {fill=black!40!cyan, text=white}}

\newcommand{\aiplans@body}[1]{
  \tikzset{aiplans action body/.style = {#1}}
}

\newcommand{\aiplans@limb}[1]{
  \tikzset{aiplans action limb/.style = {#1}}
}

% Definition of the key-value pairs that can be used with \scheme and \action.
\pgfkeys{
  /aiplans/.is family, /aiplans,
  default/.style = {
    text = {},
    effs = {},
    pres = {},
    height = 10ex,
    width = 12em,
    pre length  = 12em,
    pre lengths = {},
    eff length  = 12em,
    eff lengths = {}},
  pre/.style = {pres = { {#1}}}, % the space is important for \foreach
  eff/.style = {effs = { {#1}}},
  limb length/.style = {pre length = #1, eff length = #1},
  % text/.estore in = \aiplans@text, % probelematic for \textbf \emph etc.
  text/.store in = \aiplans@text,
  pres/.store in = \aiplans@pres,
  effs/.store in = \aiplans@effs,
  height/.estore in = \aiplans@height,
  width/.estore in = \aiplans@width,
  pre length/.estore in = \aiplans@prelength,
  pre lengths/.estore in = \aiplans@prelengths,
  eff length/.estore in = \aiplans@efflength,
  eff lengths/.estore in = \aiplans@efflengths,
}

\pgfkeys{
  /aiplans,
  body/.code = {\aiplans@body{#1}},
  limb/.code = {\aiplans@limb{#1}}
}

% \scheme defines a parameterized scheme that can be instantiated
% with \action.
% #1 name
% #2 number of arguments
% #3 key-value pairs
\newcommand{\scheme}[3]{
  \pgfkeys{/aiplans, #1/.style n args =#2{#3}}
}

% #1 from (coordinate)
% #2 to (coordinate)
% #3 tikz edge/arrow
\newcommand{\link}[3]{
  \node (aiplans circle from) at (#1) [circle,fill,inner sep=0.1em] {};
  \node (aiplans circle to)   at (#2) [circle,fill,inner sep=0.1em] {};
  \draw (aiplans circle from) #3 (aiplans circle to);
}

% #1 name
% #2 key-value pairs
\def\action#1#2{
  \def\aiplans@name{#1}

  \pgfkeys{/aiplans, default, #2}

  \setcounter{aiplans@numpres}{0}
  \foreach \x in \aiplans@pres {
    \stepcounter{aiplans@numpres}
  }

  \setcounter{aiplans@numeffs}{0}
  \foreach \x in \aiplans@effs {
    \stepcounter{aiplans@numeffs}
  }

  \node [
    minimum height = \aiplans@height,
    minimum width  = \aiplans@width,
    aiplans action body, draw, rectangle
  ] (\aiplans@name) {\aiplans@text};

  \foreach \eff [count=\k] in \aiplans@effs {
    \tikzmath{
      \s = 1/\value{aiplans@numeffs};
      \r = \s * (\k-1) + \s/2;
    }

    \coordinate (\aiplans@name/eff/start/\k) at
      ($(\aiplans@name.north east)!\r!(\aiplans@name.south east)$);

    \def\aiplans@length{\aiplans@efflength}
    \foreach \x / \y in \aiplans@efflengths {
      \if\x\k
        \xdef\aiplans@length{\y}
      \fi
    }

    \coordinate (\aiplans@name/eff/\k) at
      ($(\aiplans@name/eff/start/\k)+(\aiplans@length,0)$);

    \path[
      aiplans action limb,
      postaction = {decorate},
      decoration = {
        raise = .25 ex,
        text along path,
        text = {{\phantom i}\eff},
        text align = left
      }
    ] (\aiplans@name/eff/start/\k) -- (\aiplans@name/eff/\k);
  }

  \foreach \pre [count=\k] in \aiplans@pres{
    \tikzmath{
      \s = 1/\value{aiplans@numpres};
      \r = \s * (\k-1) + \s/2;
    }

    \coordinate (\aiplans@name/pre/start/\k) at
      ($(\aiplans@name.north west)!\r!(\aiplans@name.south west)$);

    \def\aiplans@length{\aiplans@prelength}
    \foreach \x / \y in \aiplans@prelengths {
      \if\x\k
        \xdef\aiplans@length{\y}
      \fi
    }

    \coordinate (\aiplans@name/pre/\k) at
      ($(\aiplans@name/pre/start/\k)+(-\aiplans@length,0)$);

    \path[
      aiplans action limb,
      postaction = {decorate},
      decoration = {
        raise = .25 ex,
        text along path,
        text = {\pre{\phantom i}},
        text align = right
      }
    ] (\aiplans@name/pre/\k) -- (\aiplans@name/pre/start/\k);
  }
}

% \stage draws a start and a goal action.
% #1 center (optional!)
% #2 height
% #3 distance from center
% #4 start parameters
% #5 goal parameters
% #6 name of the start action
% #7 name of the goal action
\newcommand{\stage}[7][{{(0,0)}}]{
  \action{init}{text = {\rotatebox{90}{\textbf{#6}}},
    body = {aiplans emph, left = #3 of #1},
    height = #2,
    width  = 1em,
    #4,
  }
  \action{goal}{text = {\rotatebox{90}{\textbf{#7}}},
    body = {aiplans emph, right = #3 of #1},
    height = #2,
    width  = 1em,
    #5,
  }
}

\newcommand{\ordering}[2]{
  \draw[bend left,->] (#1) to node[midway,above] {$<$} (#2);
}

\endinput