\chapter{Evaluation Makespan Optimization}

In this chapter the effectiveness of the optimization strategy from Chapter 5 will be evaluated.
The metrics that have been used are similar to the metrics in Chapter 3.
Plan step related metrics will not be featured because the amount of plan steps is not affected 
by the optimizations of Chapter 5.
The set of tested plans were selected similarly to the second test set of Chapter 3 which vary in size
from $<$10 up to 50 plan steps. In total 7429 plans have been tested.
The added makespan of all plans is 79530. 
Table 6.1 provides the compact results for all plans, while in table 6.2 the tested plans are sorted by their size.


\begin{table}[h]
    \begin{tabular}{ll}
    \hline
    Plans                   & 7429    \\ \hline
    Optimized Plans         & 2198    \\
    Optimization Ratio      & 0.295   \\
    Optimized Makespan      & 3454    \\
    Makespan Ratio          & 0.957   \\
    average optimization    & 0.888   \\
    Computation Time in min & 922.110
    \end{tabular}
    \caption{Optimization results for all 7429 plans}
\end{table}

In comparison, the makespan ratio is worse than all makespan ratios from the justification algorithms.
Also, the optimization ratio is relatively low but is able to beat the results of the well justification algorithm on POCL plans.
The average makespan optimization is on the same level as those of the justification algorithm optimizations.
Therefore, a low percentage of plans have been optimized, but the optimizations that have been made are significant.
The computation time is really high, in comparison the slowest justification algorithm (greedy justification on POCL plans) 
required for a similar amount of plans only 5 minutes.

\begin{table}
    \begin{tabular}{llllll}
    \cline{1-6}
    Plan size                    & \textgreater{}0 & \textgreater{}10 & \textgreater{}20 & \textgreater{}30 & \textgreater{}40 \\ \cline{1-6}
    Plans                         & 1921            & 1547             & 2054             & 911              & 996              \\
    Optimized problems            & 56              & 215              & 1010             & 431              & 507              \\
    Optimized plan ratio          & 0.029           & 0.139            & 0.492            & 0.473            & 0.509            \\
    Optimized makespan            & 72              & 231              & 1353             & 740              & 1126             \\
    Average makespan optimization & 0.748           & 0.887            & 0.883            & 0.892            & 0.875            \\
    Makespan ratio                & 0.993           & 0.985            & 0.937            & 0.944            & 0.937            \\
    Computation time in min       & 85.855          & 78.549           & 156.861          & 68.005           & 473.858         
    \end{tabular}
    \caption{Optimization results sorted by the plan size}
\end{table}

Regarding table 6.2 the most significant change throughout the size of the tested plans is the optimized plan ratio. 
The bigger the plans get the more likely an optimization can be found.
Although the average makespan optimization seems to get worse for plans with a size \textgreater{}10, in total more makespan is getting optimized on the bigger plans. 
Those differences are most significant for plans with a size \textless{}10 plan steps.
Only 2.9$\%$ of those plans were optimized with an average makespan optimization of 74.8$\%$.
74.8$\%$ seems like a really high average optimization, but we have to consider on a plan size \textless{}10 the removal of only one plan step 
will yield a makespan optimization of at least 90$\%$. 
For plans that have more than 20 plan steps the results have rather insignificant variations. 

Overall justification algorithms provide better makespan optimization in less computation time.
On top of that, justification algorithms have the ability to remove plan steps.
Therefore, justification algorithms are more effective for optimizing plans.
Although it has to be considered that both methods create different kinds of optimization
which means they can be used consecutively on the same plans. 